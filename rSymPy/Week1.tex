% Options for packages loaded elsewhere
\PassOptionsToPackage{unicode}{hyperref}
\PassOptionsToPackage{hyphens}{url}
%
\documentclass[
]{article}
\usepackage{amsmath,amssymb}
\usepackage{lmodern}
\usepackage{ifxetex,ifluatex}
\ifnum 0\ifxetex 1\fi\ifluatex 1\fi=0 % if pdftex
  \usepackage[T1]{fontenc}
  \usepackage[utf8]{inputenc}
  \usepackage{textcomp} % provide euro and other symbols
\else % if luatex or xetex
  \usepackage{unicode-math}
  \defaultfontfeatures{Scale=MatchLowercase}
  \defaultfontfeatures[\rmfamily]{Ligatures=TeX,Scale=1}
\fi
% Use upquote if available, for straight quotes in verbatim environments
\IfFileExists{upquote.sty}{\usepackage{upquote}}{}
\IfFileExists{microtype.sty}{% use microtype if available
  \usepackage[]{microtype}
  \UseMicrotypeSet[protrusion]{basicmath} % disable protrusion for tt fonts
}{}
\makeatletter
\@ifundefined{KOMAClassName}{% if non-KOMA class
  \IfFileExists{parskip.sty}{%
    \usepackage{parskip}
  }{% else
    \setlength{\parindent}{0pt}
    \setlength{\parskip}{6pt plus 2pt minus 1pt}}
}{% if KOMA class
  \KOMAoptions{parskip=half}}
\makeatother
\usepackage{xcolor}
\IfFileExists{xurl.sty}{\usepackage{xurl}}{} % add URL line breaks if available
\IfFileExists{bookmark.sty}{\usepackage{bookmark}}{\usepackage{hyperref}}
\hypersetup{
  pdftitle={Week 1 - Donny Lofland},
  hidelinks,
  pdfcreator={LaTeX via pandoc}}
\urlstyle{same} % disable monospaced font for URLs
\usepackage[margin=1in]{geometry}
\usepackage{color}
\usepackage{fancyvrb}
\newcommand{\VerbBar}{|}
\newcommand{\VERB}{\Verb[commandchars=\\\{\}]}
\DefineVerbatimEnvironment{Highlighting}{Verbatim}{commandchars=\\\{\}}
% Add ',fontsize=\small' for more characters per line
\usepackage{framed}
\definecolor{shadecolor}{RGB}{248,248,248}
\newenvironment{Shaded}{\begin{snugshade}}{\end{snugshade}}
\newcommand{\AlertTok}[1]{\textcolor[rgb]{0.94,0.16,0.16}{#1}}
\newcommand{\AnnotationTok}[1]{\textcolor[rgb]{0.56,0.35,0.01}{\textbf{\textit{#1}}}}
\newcommand{\AttributeTok}[1]{\textcolor[rgb]{0.77,0.63,0.00}{#1}}
\newcommand{\BaseNTok}[1]{\textcolor[rgb]{0.00,0.00,0.81}{#1}}
\newcommand{\BuiltInTok}[1]{#1}
\newcommand{\CharTok}[1]{\textcolor[rgb]{0.31,0.60,0.02}{#1}}
\newcommand{\CommentTok}[1]{\textcolor[rgb]{0.56,0.35,0.01}{\textit{#1}}}
\newcommand{\CommentVarTok}[1]{\textcolor[rgb]{0.56,0.35,0.01}{\textbf{\textit{#1}}}}
\newcommand{\ConstantTok}[1]{\textcolor[rgb]{0.00,0.00,0.00}{#1}}
\newcommand{\ControlFlowTok}[1]{\textcolor[rgb]{0.13,0.29,0.53}{\textbf{#1}}}
\newcommand{\DataTypeTok}[1]{\textcolor[rgb]{0.13,0.29,0.53}{#1}}
\newcommand{\DecValTok}[1]{\textcolor[rgb]{0.00,0.00,0.81}{#1}}
\newcommand{\DocumentationTok}[1]{\textcolor[rgb]{0.56,0.35,0.01}{\textbf{\textit{#1}}}}
\newcommand{\ErrorTok}[1]{\textcolor[rgb]{0.64,0.00,0.00}{\textbf{#1}}}
\newcommand{\ExtensionTok}[1]{#1}
\newcommand{\FloatTok}[1]{\textcolor[rgb]{0.00,0.00,0.81}{#1}}
\newcommand{\FunctionTok}[1]{\textcolor[rgb]{0.00,0.00,0.00}{#1}}
\newcommand{\ImportTok}[1]{#1}
\newcommand{\InformationTok}[1]{\textcolor[rgb]{0.56,0.35,0.01}{\textbf{\textit{#1}}}}
\newcommand{\KeywordTok}[1]{\textcolor[rgb]{0.13,0.29,0.53}{\textbf{#1}}}
\newcommand{\NormalTok}[1]{#1}
\newcommand{\OperatorTok}[1]{\textcolor[rgb]{0.81,0.36,0.00}{\textbf{#1}}}
\newcommand{\OtherTok}[1]{\textcolor[rgb]{0.56,0.35,0.01}{#1}}
\newcommand{\PreprocessorTok}[1]{\textcolor[rgb]{0.56,0.35,0.01}{\textit{#1}}}
\newcommand{\RegionMarkerTok}[1]{#1}
\newcommand{\SpecialCharTok}[1]{\textcolor[rgb]{0.00,0.00,0.00}{#1}}
\newcommand{\SpecialStringTok}[1]{\textcolor[rgb]{0.31,0.60,0.02}{#1}}
\newcommand{\StringTok}[1]{\textcolor[rgb]{0.31,0.60,0.02}{#1}}
\newcommand{\VariableTok}[1]{\textcolor[rgb]{0.00,0.00,0.00}{#1}}
\newcommand{\VerbatimStringTok}[1]{\textcolor[rgb]{0.31,0.60,0.02}{#1}}
\newcommand{\WarningTok}[1]{\textcolor[rgb]{0.56,0.35,0.01}{\textbf{\textit{#1}}}}
\usepackage{graphicx}
\makeatletter
\def\maxwidth{\ifdim\Gin@nat@width>\linewidth\linewidth\else\Gin@nat@width\fi}
\def\maxheight{\ifdim\Gin@nat@height>\textheight\textheight\else\Gin@nat@height\fi}
\makeatother
% Scale images if necessary, so that they will not overflow the page
% margins by default, and it is still possible to overwrite the defaults
% using explicit options in \includegraphics[width, height, ...]{}
\setkeys{Gin}{width=\maxwidth,height=\maxheight,keepaspectratio}
% Set default figure placement to htbp
\makeatletter
\def\fps@figure{htbp}
\makeatother
\setlength{\emergencystretch}{3em} % prevent overfull lines
\providecommand{\tightlist}{%
  \setlength{\itemsep}{0pt}\setlength{\parskip}{0pt}}
\setcounter{secnumdepth}{-\maxdimen} % remove section numbering
\ifluatex
  \usepackage{selnolig}  % disable illegal ligatures
\fi

\title{Week 1 - Donny Lofland}
\author{}
\date{\vspace{-2.5em}}

\begin{document}
\maketitle

\begin{Shaded}
\begin{Highlighting}[]
\FunctionTok{library}\NormalTok{(Deriv)}
\FunctionTok{library}\NormalTok{(mosaic)}
\end{Highlighting}
\end{Shaded}

\begin{verbatim}
## Registered S3 method overwritten by 'mosaic':
##   method                           from   
##   fortify.SpatialPolygonsDataFrame ggplot2
\end{verbatim}

\begin{verbatim}
## 
## The 'mosaic' package masks several functions from core packages in order to add 
## additional features.  The original behavior of these functions should not be affected by this.
\end{verbatim}

\begin{verbatim}
## 
## Attaching package: 'mosaic'
\end{verbatim}

\begin{verbatim}
## The following objects are masked from 'package:dplyr':
## 
##     count, do, tally
\end{verbatim}

\begin{verbatim}
## The following object is masked from 'package:Matrix':
## 
##     mean
\end{verbatim}

\begin{verbatim}
## The following object is masked from 'package:ggplot2':
## 
##     stat
\end{verbatim}

\begin{verbatim}
## The following objects are masked from 'package:stats':
## 
##     binom.test, cor, cor.test, cov, fivenum, IQR, median, prop.test,
##     quantile, sd, t.test, var
\end{verbatim}

\begin{verbatim}
## The following objects are masked from 'package:base':
## 
##     max, mean, min, prod, range, sample, sum
\end{verbatim}

\begin{Shaded}
\begin{Highlighting}[]
\FunctionTok{library}\NormalTok{(mosaicCalc)}
\end{Highlighting}
\end{Shaded}

\begin{verbatim}
## Loading required package: mosaicCore
\end{verbatim}

\begin{verbatim}
## 
## Attaching package: 'mosaicCore'
\end{verbatim}

\begin{verbatim}
## The following objects are masked from 'package:dplyr':
## 
##     count, tally
\end{verbatim}

\begin{verbatim}
## 
## Attaching package: 'mosaicCalc'
\end{verbatim}

\begin{verbatim}
## The following object is masked from 'package:stats':
## 
##     D
\end{verbatim}

\begin{Shaded}
\begin{Highlighting}[]
\FunctionTok{library}\NormalTok{(rSymPy)}
\end{Highlighting}
\end{Shaded}

\begin{verbatim}
## Loading required package: rJython
\end{verbatim}

\begin{verbatim}
## Loading required package: rJava
\end{verbatim}

\begin{verbatim}
## Loading required package: rjson
\end{verbatim}

\begin{verbatim}
## 
## Attaching package: 'rSymPy'
\end{verbatim}

\begin{verbatim}
## The following object is masked from 'package:Matrix':
## 
##     Matrix
\end{verbatim}

Find the derivatives with the respect to x of the following.

\begin{enumerate}
\def\labelenumi{\arabic{enumi}.}
\tightlist
\item
  F(x\textbar x≥0)=\(1−e^{(−\lambda x)}\)
\end{enumerate}

\begin{Shaded}
\begin{Highlighting}[]
\NormalTok{myf1}\OtherTok{=}\ControlFlowTok{function}\NormalTok{(x)\{}\DecValTok{1}\SpecialCharTok{{-}}\FunctionTok{exp}\NormalTok{(}\SpecialCharTok{{-}}\NormalTok{lambda}\SpecialCharTok{*}\NormalTok{x)\}}
\FunctionTok{print}\NormalTok{(}\FunctionTok{Deriv}\NormalTok{(myf1))}
\end{Highlighting}
\end{Shaded}

\begin{verbatim}
## function (x) 
## lambda * exp(-(lambda * x))
\end{verbatim}

\begin{center}\rule{0.5\linewidth}{0.5pt}\end{center}

\begin{enumerate}
\def\labelenumi{\arabic{enumi}.}
\setcounter{enumi}{1}
\tightlist
\item
  \(F(x|b>a)=\frac{(x−a)}{(b−a)}\)
\end{enumerate}

\begin{Shaded}
\begin{Highlighting}[]
\NormalTok{myf2 }\OtherTok{=} \ControlFlowTok{function}\NormalTok{(x)\{(x}\SpecialCharTok{{-}}\NormalTok{a)}\SpecialCharTok{/}\NormalTok{(b}\SpecialCharTok{{-}}\NormalTok{a)\}}
\FunctionTok{print}\NormalTok{(}\FunctionTok{Deriv}\NormalTok{(myf2))}
\end{Highlighting}
\end{Shaded}

\begin{verbatim}
## function (x) 
## 1/(b - a)
\end{verbatim}

\begin{center}\rule{0.5\linewidth}{0.5pt}\end{center}

\begin{enumerate}
\def\labelenumi{\arabic{enumi}.}
\setcounter{enumi}{2}
\tightlist
\item
  \(F(x|a<x≤c≤b)=\frac{(x−a)^2}{(b−a)(c−a)}\)
\end{enumerate}

\begin{Shaded}
\begin{Highlighting}[]
\NormalTok{myf3 }\OtherTok{=} \ControlFlowTok{function}\NormalTok{(x)\{(x}\SpecialCharTok{{-}}\NormalTok{a)}\SpecialCharTok{\^{}}\DecValTok{2}\SpecialCharTok{/}\NormalTok{((b}\SpecialCharTok{{-}}\NormalTok{a)(c}\SpecialCharTok{{-}}\NormalTok{a))\}}
\FunctionTok{print}\NormalTok{(}\FunctionTok{Deriv}\NormalTok{(myf3))}
\end{Highlighting}
\end{Shaded}

\begin{verbatim}
## Warning in if (che1 == "stop") {: the condition has length > 1 and only the
## first element will be used
\end{verbatim}

\begin{verbatim}
## Warning in if (che1 != "[") {: the condition has length > 1 and only the first
## element will be used
\end{verbatim}

\begin{verbatim}
## Warning in if (stch == "function") {: the condition has length > 1 and only the
## first element will be used
\end{verbatim}

\begin{verbatim}
## function (x) 
## 2 * ((x - a)/(b - a)(c - a))
\end{verbatim}

\begin{center}\rule{0.5\linewidth}{0.5pt}\end{center}

\begin{enumerate}
\def\labelenumi{\arabic{enumi}.}
\setcounter{enumi}{3}
\tightlist
\item
  \(F(x|a≤c<x<b)=1−\frac{(b−x)^2}{(b−a)(c−a)}\)
\end{enumerate}

\begin{Shaded}
\begin{Highlighting}[]
\NormalTok{myf4 }\OtherTok{=} \ControlFlowTok{function}\NormalTok{(x)}\DecValTok{1}\SpecialCharTok{{-}}\NormalTok{(b}\SpecialCharTok{{-}}\NormalTok{x)}\SpecialCharTok{\^{}}\DecValTok{2}\SpecialCharTok{/}\NormalTok{((b}\SpecialCharTok{{-}}\NormalTok{a)(c}\SpecialCharTok{{-}}\NormalTok{a))}
\FunctionTok{print}\NormalTok{(}\FunctionTok{Deriv}\NormalTok{(myf4))}
\end{Highlighting}
\end{Shaded}

\begin{verbatim}
## Warning in if (che1 == "stop") {: the condition has length > 1 and only the
## first element will be used
\end{verbatim}

\begin{verbatim}
## Warning in if (che1 != "[") {: the condition has length > 1 and only the first
## element will be used
\end{verbatim}

\begin{verbatim}
## Warning in if (stch == "function") {: the condition has length > 1 and only the
## first element will be used
\end{verbatim}

\begin{verbatim}
## function (x) 
## 2 * ((b - x)/(b - a)(c - a))
\end{verbatim}

\begin{center}\rule{0.5\linewidth}{0.5pt}\end{center}

Solve the following definite and indefinite integrals

\begin{enumerate}
\def\labelenumi{\arabic{enumi}.}
\setcounter{enumi}{4}
\tightlist
\item
  \(\int_0^{10} 3x^3dx\)
\end{enumerate}

\begin{Shaded}
\begin{Highlighting}[]
\NormalTok{myf5}\OtherTok{=}\ControlFlowTok{function}\NormalTok{(x)}\DecValTok{3}\SpecialCharTok{*}\NormalTok{x}\SpecialCharTok{\^{}}\DecValTok{3}
\FunctionTok{integrate}\NormalTok{(}\FunctionTok{Vectorize}\NormalTok{(myf5),}\DecValTok{0}\NormalTok{,}\DecValTok{10}\NormalTok{)}
\end{Highlighting}
\end{Shaded}

\begin{verbatim}
## 7500 with absolute error < 8.3e-11
\end{verbatim}

\begin{center}\rule{0.5\linewidth}{0.5pt}\end{center}

\begin{enumerate}
\def\labelenumi{\arabic{enumi}.}
\setcounter{enumi}{5}
\tightlist
\item
  \(\int_0^x x\lambda e^{−\lambda x}dx\)
\end{enumerate}

Solution:

Substitution \(u=-\lambda x\); \(du=-\lambda dx\);
\(dx=\frac{du}{-\lambda}\);

\(\int -u e^u \frac{du}{-\lambda}\);

\(\frac{1}{\lambda}\int u e^u du\)

solve \(\int u e^u du\) by Parts: \(s=u; ds=du\); \(dv=e^udu; v=e^u+C\);

formula: \(sv-\int vds\)

\(=u e^u - \int e^u du\)

\(=u e^u - e^u\)

\(= (u-1)e^u\)

plugging back in: \(\frac{1}{\lambda}(u-1)e^u\)

replacing \(u=-\lambda x\) back in: \(\frac{1}{\lambda}\int u e^u du\)

\(=\frac{1}{\lambda}(-\lambda x - 1)e^{-\lambda x}\)

\(= -\frac{(\lambda x + 1) e^{-\lambda x}}{\lambda}\)

lastly we need to integrate over \(0\to x\):

\(-\frac{(\lambda x + 1) e^{-\lambda x}}{\lambda} ]_0^x\)
\(=-\frac{(\lambda x + 1) e^{-\lambda x}}{\lambda} - \frac{1}{\lambda}\)
\(=-\frac{(\lambda x + 1) e^{-\lambda x}+1}{\lambda}\)

\begin{Shaded}
\begin{Highlighting}[]
\FunctionTok{library}\NormalTok{(rSymPy)}
\FunctionTok{sympy}\NormalTok{(}\StringTok{"x = Symbol(\textquotesingle{}x\textquotesingle{})"}\NormalTok{)}
\end{Highlighting}
\end{Shaded}

\begin{verbatim}
## [1] "x"
\end{verbatim}

\begin{Shaded}
\begin{Highlighting}[]
\FunctionTok{sympy}\NormalTok{(}\StringTok{"l = Symbol(\textquotesingle{}lambda\textquotesingle{})"}\NormalTok{)}
\end{Highlighting}
\end{Shaded}

\begin{verbatim}
## [1] "lambda"
\end{verbatim}

\begin{Shaded}
\begin{Highlighting}[]
\FunctionTok{sympy}\NormalTok{(}\StringTok{"integrate(x*l*exp({-}l*x),(x,0,x))"}\NormalTok{)}
\end{Highlighting}
\end{Shaded}

\begin{verbatim}
## [1] "lambda*(-exp(-lambda*x)/lambda**2 - x*exp(-lambda*x)/lambda) + 1/lambda"
\end{verbatim}

Answer from sympy:
\(\frac{-\lambda e^{-\lambda x}}{\lambda^2}-\frac{x e^{-\lambda x}}{\lambda}+\frac{1}{\lambda}\)

I'm not sure how to get r to calculate this or it was intended that we
to this by hand. I made my best stab at doing by hand and code in latex
for the notebook.

\begin{center}\rule{0.5\linewidth}{0.5pt}\end{center}

\begin{enumerate}
\def\labelenumi{\arabic{enumi}.}
\setcounter{enumi}{6}
\tightlist
\item
  \(\int_0^{0.5} \frac{1}{b−a}dx\)
\end{enumerate}

Solution: \(\frac{1}{b-a} x ]_0^x = \frac{x}{b-a} - 0 = \frac{x}{b-a}\)

\begin{Shaded}
\begin{Highlighting}[]
\FunctionTok{library}\NormalTok{(rSymPy)}
\FunctionTok{sympy}\NormalTok{(}\StringTok{"x = Symbol(\textquotesingle{}x\textquotesingle{})"}\NormalTok{)}
\end{Highlighting}
\end{Shaded}

\begin{verbatim}
## [1] "x"
\end{verbatim}

\begin{Shaded}
\begin{Highlighting}[]
\FunctionTok{sympy}\NormalTok{(}\StringTok{"a = Symbol(\textquotesingle{}alpha\textquotesingle{})"}\NormalTok{)}
\end{Highlighting}
\end{Shaded}

\begin{verbatim}
## [1] "alpha"
\end{verbatim}

\begin{Shaded}
\begin{Highlighting}[]
\FunctionTok{sympy}\NormalTok{(}\StringTok{"B = Symbol(\textquotesingle{}beta\textquotesingle{})"}\NormalTok{)}
\end{Highlighting}
\end{Shaded}

\begin{verbatim}
## [1] "beta"
\end{verbatim}

\begin{Shaded}
\begin{Highlighting}[]
\FunctionTok{sympy}\NormalTok{(}\StringTok{"integrate(1/(B{-}a),(x,0,x))"}\NormalTok{)}
\end{Highlighting}
\end{Shaded}

\begin{verbatim}
## [1] "x/(beta - alpha)"
\end{verbatim}

\(\frac{x}{b-a}\)

I'm not sure how to get r to calculate this or it was intended that we
to this by hand. I'm a python guy so brought over sympy to help. I made
my best stab at doing by hand and code in latex for the notebook.

\begin{center}\rule{0.5\linewidth}{0.5pt}\end{center}

\begin{enumerate}
\def\labelenumi{\arabic{enumi}.}
\setcounter{enumi}{7}
\tightlist
\item
  \(\int_0^x x \frac{1}{\Gamma(\alpha)\beta^\alpha} x^{\alpha -1} e^{-\beta x} dx\)
\end{enumerate}

Solution:

\begin{Shaded}
\begin{Highlighting}[]
\FunctionTok{library}\NormalTok{(rSymPy)}
\FunctionTok{sympy}\NormalTok{(}\StringTok{"x = Symbol(\textquotesingle{}x\textquotesingle{})"}\NormalTok{)}
\end{Highlighting}
\end{Shaded}

\begin{verbatim}
## [1] "x"
\end{verbatim}

\begin{Shaded}
\begin{Highlighting}[]
\FunctionTok{sympy}\NormalTok{(}\StringTok{"a = Symbol(\textquotesingle{}alpha\textquotesingle{})"}\NormalTok{)}
\end{Highlighting}
\end{Shaded}

\begin{verbatim}
## [1] "alpha"
\end{verbatim}

\begin{Shaded}
\begin{Highlighting}[]
\FunctionTok{sympy}\NormalTok{(}\StringTok{"B = Symbol(\textquotesingle{}beta\textquotesingle{})"}\NormalTok{)}
\end{Highlighting}
\end{Shaded}

\begin{verbatim}
## [1] "beta"
\end{verbatim}

\begin{Shaded}
\begin{Highlighting}[]
\FunctionTok{sympy}\NormalTok{(}\StringTok{"G = Symbol(\textquotesingle{}Gamma\textquotesingle{})"}\NormalTok{)}
\end{Highlighting}
\end{Shaded}

\begin{verbatim}
## [1] "Gamma"
\end{verbatim}

\begin{Shaded}
\begin{Highlighting}[]
\FunctionTok{sympy}\NormalTok{(}\StringTok{"integrate(x * (G*a*B**a)**({-}1) * x**(a) * exp({-}B*x), (x,0,x))"}\NormalTok{)}
\end{Highlighting}
\end{Shaded}

\begin{verbatim}
## [1] "Integral(x*beta**(-alpha)*x**alpha*exp(-beta*x)/(Gamma*alpha), (x, 0, x))"
\end{verbatim}

sympy solution:
\(\frac{x \beta^{-\alpha}x^{\alpha-1}e^{-\beta x}}{\Gamma \alpha}]_0^x\)

I really have no idea how to tackle this or was was hinted by the gamma
function. I'm hoping we can cover this at some point so I can understand
exactly how to approach the problem. Did you want us to hand solve this,
use r or some combination? I had assumed this was a r task since we were
using R to solve the early problems. Anyways, for problem 7 \& 8, I
could certainly use some insights. Thanks! Donny

Hint: the last part of the equation is beginning with the gamma function
is a Gamma probability distribution function. Try rearranging the terms
to integrate another Gamma distribution out of the integral, as pdfs
must integrate to 1.

\begin{center}\rule{0.5\linewidth}{0.5pt}\end{center}

With the following matrix,
\(X = \begin{bmatrix}1 & 2 & 3 \\ 3 & 3 & 1 \\ 4 & 6 & 8\end{bmatrix}\)

\begin{Shaded}
\begin{Highlighting}[]
\NormalTok{myMatrix }\OtherTok{\textless{}{-}} \FunctionTok{matrix}\NormalTok{(}\FunctionTok{c}\NormalTok{(}\DecValTok{1}\NormalTok{,}\DecValTok{3}\NormalTok{,}\DecValTok{4}\NormalTok{,}\DecValTok{2}\NormalTok{,}\DecValTok{3}\NormalTok{,}\DecValTok{6}\NormalTok{,}\DecValTok{3}\NormalTok{,}\DecValTok{1}\NormalTok{,}\DecValTok{8}\NormalTok{),}\DecValTok{3}\NormalTok{,}\DecValTok{3}\NormalTok{)}
\NormalTok{myMatrix}
\end{Highlighting}
\end{Shaded}

\begin{verbatim}
##      [,1] [,2] [,3]
## [1,]    1    2    3
## [2,]    3    3    1
## [3,]    4    6    8
\end{verbatim}

\begin{center}\rule{0.5\linewidth}{0.5pt}\end{center}

\begin{enumerate}
\def\labelenumi{\arabic{enumi}.}
\setcounter{enumi}{8}
\tightlist
\item
  Invert it using Gaussian row reduction.
\end{enumerate}

\begin{Shaded}
\begin{Highlighting}[]
\FunctionTok{library}\NormalTok{(matlib)}
\end{Highlighting}
\end{Shaded}

\begin{verbatim}
## 
## Attaching package: 'matlib'
\end{verbatim}

\begin{verbatim}
## The following object is masked from 'package:rJava':
## 
##     J
\end{verbatim}

\begin{Shaded}
\begin{Highlighting}[]
\FunctionTok{gaussianElimination}\NormalTok{(myMatrix, }\FunctionTok{numeric}\NormalTok{(}\DecValTok{3}\NormalTok{))}
\end{Highlighting}
\end{Shaded}

\begin{verbatim}
##      [,1] [,2] [,3] [,4]
## [1,]    1    0    0    0
## [2,]    0    1    0    0
## [3,]    0    0    1    0
\end{verbatim}

\begin{Shaded}
\begin{Highlighting}[]
\FunctionTok{gaussianElimination}\NormalTok{(myMatrix, }\FunctionTok{diag}\NormalTok{(}\DecValTok{3}\NormalTok{))}
\end{Highlighting}
\end{Shaded}

\begin{verbatim}
##      [,1] [,2] [,3] [,4] [,5]  [,6]
## [1,]    1    0    0 -4.5 -0.5  1.75
## [2,]    0    1    0  5.0  1.0 -2.00
## [3,]    0    0    1 -1.5 -0.5  0.75
\end{verbatim}

\begin{Shaded}
\begin{Highlighting}[]
\FunctionTok{inv}\NormalTok{(myMatrix)}
\end{Highlighting}
\end{Shaded}

\begin{verbatim}
##      [,1] [,2]  [,3]
## [1,] -4.5 -0.5  1.75
## [2,]  5.0  1.0 -2.00
## [3,] -1.5 -0.5  0.75
\end{verbatim}

\begin{center}\rule{0.5\linewidth}{0.5pt}\end{center}

\begin{enumerate}
\def\labelenumi{\arabic{enumi}.}
\setcounter{enumi}{9}
\tightlist
\item
  Find the determinant.
\end{enumerate}

\begin{Shaded}
\begin{Highlighting}[]
\FunctionTok{det}\NormalTok{(myMatrix)}
\end{Highlighting}
\end{Shaded}

\begin{verbatim}
## [1] -4
\end{verbatim}

\begin{center}\rule{0.5\linewidth}{0.5pt}\end{center}

\begin{enumerate}
\def\labelenumi{\arabic{enumi}.}
\setcounter{enumi}{10}
\tightlist
\item
  Conduct LU decomposition
\end{enumerate}

\begin{Shaded}
\begin{Highlighting}[]
\NormalTok{lum }\OtherTok{\textless{}{-}} \FunctionTok{lu}\NormalTok{(myMatrix)}
\NormalTok{elu }\OtherTok{\textless{}{-}} \FunctionTok{expand}\NormalTok{(lum)}

\NormalTok{(L }\OtherTok{\textless{}{-}}\NormalTok{ elu}\SpecialCharTok{$}\NormalTok{L)}
\end{Highlighting}
\end{Shaded}

\begin{verbatim}
## 3 x 3 Matrix of class "dtrMatrix" (unitriangular)
##      [,1]       [,2]       [,3]      
## [1,]  1.0000000          .          .
## [2,]  0.7500000  1.0000000          .
## [3,]  0.2500000 -0.3333333  1.0000000
\end{verbatim}

\begin{Shaded}
\begin{Highlighting}[]
\NormalTok{(U }\OtherTok{\textless{}{-}}\NormalTok{ elu}\SpecialCharTok{$}\NormalTok{U)}
\end{Highlighting}
\end{Shaded}

\begin{verbatim}
## 3 x 3 Matrix of class "dtrMatrix"
##      [,1]       [,2]       [,3]      
## [1,]  4.0000000  6.0000000  8.0000000
## [2,]          . -1.5000000 -5.0000000
## [3,]          .          . -0.6666667
\end{verbatim}

\begin{Shaded}
\begin{Highlighting}[]
\NormalTok{(P }\OtherTok{\textless{}{-}}\NormalTok{ elu}\SpecialCharTok{$}\NormalTok{P)}
\end{Highlighting}
\end{Shaded}

\begin{verbatim}
## 3 x 3 sparse Matrix of class "pMatrix"
##           
## [1,] . . |
## [2,] . | .
## [3,] | . .
\end{verbatim}

\begin{Shaded}
\begin{Highlighting}[]
\NormalTok{L }\SpecialCharTok{\%*\%}\NormalTok{ U}
\end{Highlighting}
\end{Shaded}

\begin{verbatim}
## 3 x 3 Matrix of class "dgeMatrix"
##      [,1] [,2] [,3]
## [1,]    4    6    8
## [2,]    3    3    1
## [3,]    1    2    3
\end{verbatim}

\begin{center}\rule{0.5\linewidth}{0.5pt}\end{center}

\begin{enumerate}
\def\labelenumi{\arabic{enumi}.}
\setcounter{enumi}{11}
\tightlist
\item
  Multiply the matrix by it's inverse.
\end{enumerate}

\begin{Shaded}
\begin{Highlighting}[]
\NormalTok{inv }\OtherTok{\textless{}{-}} \FunctionTok{round}\NormalTok{(}\FunctionTok{solve}\NormalTok{(myMatrix),}\DecValTok{2}\NormalTok{)}
\NormalTok{myMatrix}\SpecialCharTok{\%*\%}\NormalTok{inv}
\end{Highlighting}
\end{Shaded}

\begin{verbatim}
##      [,1] [,2] [,3]
## [1,]    1    0    0
## [2,]    0    1    0
## [3,]    0    0    1
\end{verbatim}

\end{document}
